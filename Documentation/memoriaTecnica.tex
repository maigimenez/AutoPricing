\documentclass[spanish,a4paper,10pt]{article}
\usepackage[utf8]{inputenc}
\usepackage[spanish]{babel}

% Parámetro para poder usar colores
\usepackage[pdftex,usenames,dvipsnames]{color}

%Paquete para justificar el texto
\usepackage{ragged2e} 

% Añado anchura al texto y elimino el margen horizontal
\addtolength{\textwidth}{45mm}
\addtolength{\textheight}{15mm}
\addtolength{\oddsidemargin}{-20mm}
\addtolength{\evensidemargin}{-20mm}
\addtolength{\parskip}{1,5mm}


% Penalizo la finalización o el inicio de la página con 
% lineas huerfanas
\clubpenalty=10000
\widowpenalty=10000

% Incluyo el paquete para gráficos
\usepackage{graphicx}
% Con esta opción no cargo las imágenes sólo sus espacios
%\usepackage[draft]{graphicx}
%\usepackage{wrapfig} %Las imágenes quedan rodeadas de texto

% Incluyo el paquete para hiperreferncias
%\usepackage{hyperref} 

%Con esto incluyo secciones en la tabla de contenidos
\setcounter{tocdepth}{3}


%opening
\title{Sistema de tarifación automática para redes vehiculares basado en smartphones Android y Google Maps }
\author{Maria Teresa Giménez Fayos}

\begin{document}

\maketitle

\begin{abstract}

\end{abstract}

\section{Proposición técnica}
  
\section{Análisis del sistema}
  \subsection{Aplicación cliente}
  Desarrollaremos una aplicación Android
  \subsection{Aplicación servidor}

\section{Requisitos}
A continuación se detallan los requisitos que el sistema implementará. Las funcionalidades se dividen segun el rol del usuario al que van diriguidos. 

\subsection{Rol cliente}

El cliente deberá poder tener acceso a las siguientes funcionalidades cuando se conecte a la aplicación servidor:
\begin{itemize}
 \item El cliente deberá ser capaz de identificarse en el sistema.
 \item Una vez autentificadado, será capaz de modificar sus datos personales.
 \item El cliente podrá ver las facturas que se han generado.
\end{itemize}

Mientras que cuando este accediendo al cliente de la aplicación deberá poder llevar a cabo las siguientes tareas:
\begin{itemize}
 \item Identificarse en el sistema.
 \item Seleccionar el vehículo que realizará las pruebas si tiene más de uno asociado a este cliente. 
 \item Seleccionar el dispositivo android que realizará las pruebas.
 \item Iniciar las pruebas. Cuando se inicien las pruebas el cliente podrá seleccionar si desea que se envíen los datos en tiempo real o 
si decdide almacenarlos en el dispositivo para después enviarlos.
 \item Finalizar las pruebas con el vehículo. Si no decidió enviar los datos en tiempo real ahora se enviarán todos, así como si quedaran 
datos pendientes de enviar porque en algún momento se hubiera perdido la conexión. 
  \item Consultar el gasto generado durante la prueba que se esta realizando. 
  \item Consultar un historial de las pruebas realizadas para el cliente que se ha identificado.
\end{itemize}

\subsection{Rol administrador}

El administrador podrá llevar a cabo las siguientes tareas en la aplicación servidor
\begin{itemize}
 \item Identificarse, de igual modo como lo hace el cliente, aunque por supuesto con un rol diferente. 
 \item Consultar, modificar, crear y eliminar datos de clientes. 
 \item Consultar, modificar, crear y eliminar datos de los vehículos. 
 \item Consultar, modificar, crear y eliminar pistas de pruebas. Seleccionando el área del mapa que corresponde a la pista de prueba, el administrador creará nuevas pistas de pruebas. 
Cuando se cree una pista nueva, el administrador asociará una tarifa existente a la nueva pista o creará una nueva tarifa. 
 \item Consultar y modificar tarifas asociadas a las pistas de pruebas.
 \item Consultar, modificar, crear y eliminar dispositivos android para realizar las pruebas. 
 \item Vincular y desvincular un dispositivo android con un vehículo.
 \item Consultar el historial de las pruebas realizadas. 
 \item Consultar el hisotorial de facturas.
 \item Generar facturas, agrupando pruebas realizadas. 
\end{itemize}

La funcionalidades que puede realizar el administrador en la aplicación cliente, serán las mismas que puede realizar el cliente incluyendo la posibilidad de vincular un terminal con un cliente para realizar 
las pruebas. 




\section{Diseño}
\subsection{Aplicación cliente}
\subsection{Aplicación servidor}

\section{Implementación}
\subsection{Aplicación cliente}
\subsection{Aplicación servidor}

\section{Pruebas}
\subsection{Aplicación cliente}
\subsection{Aplicación servidor}


\end{document}
